\documentclass{article}
\usepackage{chngpage}
\usepackage[utf8]{inputenc}
\usepackage[margin=2cm]{geometry}
\usepackage{tikz}

\title{Assignment 1: Mandelbrot Fractal}
\author{Julien GAUTIER}

\begin{document}
\maketitle

\section{La SIO}
La SIO (ou Symetric Image Optimization) est la plus simple des optimisations. Il suffit de remarquer que la fractal
de Mandelbrot est symétrique selon l’axe des réels. Il suffit calculer la motié supérieure de l’image puis de la
dupliquer sur la partie inférieure en la retournant verticalement. En pratique, la duplication de l’image est effectuée
grâce à un appel à memcpy par ligne. Le speed-up attendu est d’environ 2 et le speed-up mesuré est de 1.9.

\section{Mettre en cache la Heat LUT}
Dans l’implémentation naive, la fonction \texttt{heat\_lut} est appelée pour chaque pixel. Mais il n’existe que $n\_iterations+1$
couleurs possibles. On peut donc facilement les mettre en cache et faire un simple look-up pour chaque pixel. Pour
$n\_iterations = 100$, il nous faut seulement 400 octets de cache, ce qui rentre aisément dans le cache L1.

\section{L’optimisation SIMD}
Avec AVX2 on peut effectuer des calculs en parallel sur 256 bits, ce qui représentent 8 floats de 32 bits. Soit Z le
nombre complexe associé au pixel à calculer. J’ai calculé sa norme au carré à l’aide des instructions \texttt{\_mm256\_mul\_ps}
et \texttt{\_mm256\_add\_ps}. Je teste si cette valeur est supérieure au \texttt{threshold} (égale à 4) avec \texttt{\_mm256\_cmp\_ps}. J’obtiens
un vecteur dont la case vaut 0 si $||z||^2 \geq 4$ et -1 si $||z||^2 < 4$. J’utilise \texttt{\_mm256\_and\_ps} pour remplacer les -1 en
1. J’incrémente \texttt{v\_iter} (mon vecteur comptant le nombre d’itérations) à l’aide du masque. De cette manière, seul
les pixels dont la norme est inférieure à 4 sont incrementés. Si toutes les normes sont supérieures à 4, il est inutile
de continuer. Si le \texttt{mask} est égal au vecteur ne comportant que des 1, on sort de la boucle. On extrait ensuite les
valeurs des itérations à l’aide de \texttt{\_mm256\_cvtps\_epi32} que l’on stocke dans l’histogramme. Le speed-up attendu est
d’environ 8 car on traite 8 pixels en simultané. Le speed-up obtenu est de 5.

\section{Le parallélisme multi-thread}
Soit N le nombre de ceurs de la machine (obtenue grâce à \texttt{std::hardware\_concurrency}). On pourrais diviser
l’image en N bandes et assigner chaque threads au calcul d’une de ces bandes. Mais cette méthode n’est pas
optimale car le thread qui doit calculer la bande supérieure de l’image terminera bien avant les autres. En effet,
beaucoup des pixels du haut de l’image divergent rapidement. Il est plus efficace d’assigner au thread numero $i$
toutes les lignes numero $i + kN$. Ainsi le travil est plus équitablement répartie entre les threads. Sur une machine
possédant 2 coeurs avec Hyperthreading, donc 4 coeurs logiques, le speed-up attendu est d’environ 4. Le speed-up
obtenu est que de 3,2. Avec 2 threads, on obtient bien un speed-up de 2,0.
\end{document}
